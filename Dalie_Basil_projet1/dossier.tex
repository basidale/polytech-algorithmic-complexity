\documentclass{report}
\usepackage[french]{babel}
\usepackage[a4paper, margin=1in]{geometry}
\usepackage{changepage}
\title{Complexité et algorithmique - Miniprojet 1}
\author{Basil dalié}
\date{Novembre 2018}
\begin{document}
\maketitle

\chapter{Introduction}
Dans le cadre du module Complexité et algorithmique inscrit dans le programme pédagogique de quatrième année en Siences Informatiques à l'école Polytech Nice-Sophia, il nous a été demandé de réaliser quatre algorithmes pour résoudre le problème de la recherche du $n\up{ème}$ bit sur une machine de turing.

Ce problème consiste à retourner, pour une donnée de longueur $N$ constitué d'un mot $m$ sur l'alphabet $\{A, B, C, D\}$ suivi de la lettre X puis d'un nombre $n$, la $n\up{ème}$ lettre du mot m ou alors X si le nombre n est supérieur à la longueur de m (e.g. si m est AABA, le résultat doit être B si n est égal à 3 et X si n est égal à 5).

Chacun de ces algorithmes devaient respecter des contraintes distinctes : pour le premier, la machine de turing dispose de deux bandes, contre une bande pour les trois suivants. Le second devait avoir une complexité de l'ordre de $O(N^2)$ et les deux suivants de $O(N\log{}N)$. Pour le troisième algorithme la donnée $n$ devait être donnée en binaire, alors qu'elle devait être donnée en unaire pour les trois autres.

Au cours de ces pages, vous pourrez trouver pour chacun des algorithmes réalisés, sa description, ainsi que la preuve et l'analyse de sa complexité

\chapter{Première solution}
\section{Description}
La première solution pour la recherche du $n\up{ème}$ bit est une machine de Turing à deux bandes, le nombre $n$ est donné en unaire. la complexité de l'algorithme est de l'ordre de $O(N)$.

\subsection{Alphabet}
Pour cette solution, l'ensemble des symboles $T$ est égal à l'union de l'ensemble des symboles de données $I$ avec l'ensemble $\{\_\}$ où le symbole $\_$ représente le symbole blanc. Ses éléments sont : $\{A, B, C, D, X, 1, \_\}$, le symbole $1$ représente l'unique chiffre pour représenter la donnée $n$ en unaire.

\subsection{Ensemble d'état}
La machine de turing comporte 11 états sans compter l'état d'erreur $@NO$.

\subsection{Fonction de transition}


\begin{adjustwidth}{-1.5cm}{0cm}
\begin{tabular}{| c | c c | c c | c c | c |}
  \hline
  état & l1 & l2 & e1 & e2 & d1 & d2 & nouvel état \\
  \hline
  MOVE\_TO\_X\_ON\_TAPE\_ONE & A & \_ & A & \_ & R & S & MOVE\_TO\_X\_ON\_TAPE\_ONE \\
       & B & \_ & B & \_ & R & S & MOVE\_TO\_X\_ON\_TAPE\_ONE \\
       & C & \_ & C & \_ & R & S & MOVE\_TO\_X\_ON\_TAPE\_ONE \\
       & D & \_ & D & \_ & R & S & MOVE\_TO\_X\_ON\_TAPE\_ONE \\
       & X & \_ & X & \_ & R & S & WRITE\_N\_ON\_TAPE\_TWO \\
  \hline
  WRITE\_N\_ON\_TAPE\_TWO & 1 & \_ & \_ & 1 & R & R & WRITE\_N\_ON\_TAPE\_TWO \\
       & \_ & \_ & \_ & \_ & L & L & MOVE\_TO\_WORD\_RIGHT\_ON\_TAPE\_ONE \\
  \hline
  MOVE\_TO\_WORD\_RIGHT\_ON\_TAPE\_ONE & \_ & 1 & \_ & 1 & L & S & MOVE\_TO\_WORD\_RIGHT\_ON\_TAPE\_ONE \\
       & X & 1 & X & 1 & L & S & MOVE\_TO\_WORD\_LEFT\_ON\_TAPE\_ONE \\
  \hline
  MOVE\_TO\_WORD\_LEFT\_ON\_TAPE\_ONE & A & 1 & A & 1 & L & S & MOVE\_TO\_WORD\_LEFT\_ON\_TAPE\_ONE \\
       & B & 1 & B & 1 & L & S & MOVE\_TO\_WORD\_LEFT\_ON\_TAPE\_ONE \\
       & C & 1 & C & 1 & L & S & MOVE\_TO\_WORD\_LEFT\_ON\_TAPE\_ONE \\
       & D & 1 & D & 1 & L & S & MOVE\_TO\_WORD\_LEFT\_ON\_TAPE\_ONE \\
       & \_ & 1 & \_ & 1 & R & S & MOVE\_TO\_WORD\_LEFT\_ON\_TAPE\_TWO \\
  \hline
  MOVE\_TO\_WORD\_LEFT\_ON\_TAPE\_TWO & A & 1 & A & 1 & S & L & MOVE\_TO\_WORD\_LEFT\_ON\_TAPE\_TWO \\
  & B & 1 & B & 1 & S & L & MOVE\_TO\_WORD\_LEFT\_ON\_TAPE\_TWO \\
  & C & 1 & C & 1 & S & L & MOVE\_TO\_WORD\_LEFT\_ON\_TAPE\_TWO \\
  & D & 1 & D & 1 & S & L & MOVE\_TO\_WORD\_LEFT\_ON\_TAPE\_TWO \\
  & A & \_ & A & \_ & S & R & ERASE\_RIGHT \\
  & B & \_ & B & \_ & S & R & ERASE\_RIGHT \\
  & C & \_ & C & \_ & S & R & ERASE\_RIGHT \\
       & D & \_ & D & \_ & S & R & ERASE\_RIGHT \\
  \hline
  ERASE\_RIGHT & A & 1 & A & 1 & R & R & ERASE\_RIGHT \\
       & B & 1 & B & 1 & R & R & ERASE\_RIGHT \\
       & C & 1 & C & 1 & R & R & ERASE\_RIGHT \\
       & D & 1 & D & 1 & R & R & ERASE\_RIGHT \\
       & A & \_ & \_ & \_ & R & R & ERASE\_RIGHT \\
       & B & \_ & \_ & \_ & R & R & ERASE\_RIGHT \\
       & C & \_ & \_ & \_ & R & R & ERASE\_RIGHT \\
       & D & \_ & \_ & \_ & R & R & ERASE\_RIGHT \\
       & X & \_ & \_ & \_ & L & L & MOVE\_TO\_NTH \\
       & X & 1 & X & 1 & L & L & ERASE\_LEFT \\
  \hline
  MOVE\_TO\_NTH & \_ & \_ & \_ & \_ & L & L & MOVE\_TO\_NTH \\
       & A & 1 & A & 1 & L & L & ERASE\_LEFT \\
       & B & 1 & B & 1 & L & L & ERASE\_LEFT \\
       & C & 1 & C & 1 & L & L & ERASE\_LEFT \\
       & D & 1 & D & 1 & L & L & ERASE\_LEFT \\
  \hline
  ERASE\_LEFT & A & 1 & \_ & 1 & L & L & ERASE\_LEFT \\
       & B & 1 & \_ & 1 & L & L & ERASE\_LEFT \\
       & C & 1 & \_ & 1 & L & L & ERASE\_LEFT \\
       & D & 1 & \_ & 1 & L & L & ERASE\_LEFT \\
       & \_ & \_ & \_ & \_ & R & R & ERASE\_TAPE\_TWO \\
  \hline
  ERASE\_TAPE\_TWO & A & 1 & A & \_ & S & R & ERASE\_TAPE\_TWO \\
       & B & 1 & B & \_ & S & R & ERASE\_TAPE\_TWO \\
       & C & 1 & C & \_ & S & R & ERASE\_TAPE\_TWO \\
       & D & 1 & D & \_ & S & R & ERASE\_TAPE\_TWO \\
       & \_ & 1 & \_ & \_ & S & R & ERASE\_TAPE\_TWO \\
       & A & \_ & A & \_ & S & L & MOVE\_TO\_NTH\_FINAL \\
       & B & \_ & B & \_ & S & L & MOVE\_TO\_NTH\_FINAL \\
       & C & \_ & C & \_ & S & L & MOVE\_TO\_NTH\_FINAL \\
       & D & \_ & D & \_ & S & L & MOVE\_TO\_NTH\_FINAL \\
  \hline
  MOVE\_TO\_NTH\_FINAL & \_ & \_ & \_ & \_ & R & S & MOVE\_TO\_NTH\_FINAL \\
       & A & \_ & A & \_ & S & S & YES \\
       & B & \_ & B & \_ & S & S & YES \\
       & C & \_ & C & \_ & S & S & YES \\
       & D & \_ & D & \_ & S & S & YES \\
  \hline
\end{tabular}
\end{adjustwidth}

\subsection{Description de l'algorithme}


\end{document}
