\chapter{Première solution}

\section{Description}
La première solution pour la recherche du $n\up{ème}$ bit est une machine de Turing à deux bandes, le nombre $n$ est donné en unaire. la complexité de l'algorithme est de l'ordre de $O(N)$.

\subsection{Alphabet}
Pour cette solution, l'ensemble des symboles $T$ est égal à l'union de l'ensemble des symboles de données $I$ avec l'ensemble $\{\_\}$ où le symbole $\_$ représente le symbole blanc. Ses éléments sont : $\{A, B, C, D, X, 1, \_\}$, le symbole $1$ représente l'unique chiffre pour représenter la donnée $n$ en unaire.

\subsection{Ensemble d'état}
La machine de turing comporte 11 états en comptant l'état final $q_f$.

\subsection{Fonction de transition}
Voir table de transition en annexe A.

\subsection{Description de l'algorithme}
L'algorithme se déroule de la manière suivante : au démarrage, la tête de lecture/écriture de la première bande se trouve sur le premier symbole du mot $m$ et celle de la seconde bande se trouve au même niveau.
A l'état initial ($q_0$), tant qu'un symbole A, B, C, ou D est lu sous la tête de lecture/écriture de la première bande, celle-ci se déplace vers la droite et la machine reste dans la même état.

Lorsque la tête de lecture/écriture de la première bande atteint le symbole X, cette dernière est déplacée vers la droite et on passe dans l'état $q_1$ (Copie du nombre $n$ sur la deuxième bande).

A partir de ce moment, tant qu'un symbole 1 est lu sous la tête de lecture/écriture de la première bande, ce symbole est effacé de la première bande et recopié sur la seconde bande et les deux têtes de lecture/écritures se déplacent vers la droite.
Une fois que tous les symboles 1 ont été consommés, la machine passe dans l'état $q_2$ (Déplacement de la tête de lecture/écriture de la première bande sur le symbole X) dans lequel la tête de lecture/écriture de la première bande se déplace vers la gauche tant qu'un symbole blanc est lu.

Lorsque le symbole X est lu par la tête de lecture/écriture de la première bande, celle-ci se déplace vers la gauche et la machine passe dans l'état $q_3$ (Déplacement de la tête de lecture/écriture de la première bande sur le premier symbole du mot $m$).

Dans cet état, tant qu'un symbole A, B, C, ou D est lu par la tête de lecture/écriture de la première bande, celle-ci se déplace vers la gauche, et lorsque le symbole blanc (\_) est lu, elle se déplace vers la droite et la machine passe dans l'état $q_4$  (Déplacement de la tête de lecture/écriture de la seconde bande en dessous du premier symbole du mot $m$).

A partir de ce moment, tant qu'un symbole 1 est lu sous la tête de lecture/écriture de la seconde bande, celle-ci se déplace vers la gauche, et lorsque le symbole blanc est lu, elle se déplace vers la droite et la machine passe dans l'état $q_5$ (effaçage des symboles situés après le symbole d'indice $n$).

A ce moment de l'algorithme, la première bande contient le mot $m$ suivi du symbole X, la seconde bande contient le nombre $n$ écrit en unaire dont le premier symbole est situé juste en dessous du premier symbole du mot $m$. Les tête de lecture/écriture des bandes 1 et 2 sont situés respectivement sur le premier symbole du mot $m$ et sur le premier symbole du nombre $n$.

A l'état $q_5$, tant que les tête de lecture/écriture des bandes 1 et 2 lisent respectivement un symbole parmi $\{A, B, C, D\}$ et un symbole 1, les deux têtes de lecture/écriture se déplacent vers la droite. Si la tête de lecture/écriture de la première bande lit un symbole parmi $\{A, B, C, D, X\}$ et la tête de lecture/écriture de la seconde bande lit un symbole blanc ($\_$) on écrit le symbole blanc sous la tête de lecture/écriture de la première bande et les deux têtes de lecture/écriture se déplacent vers la droite. Dans le cas où le symbole $X$ est lu par la tête de lecture/écriture de la première bande, si la tête de lecture/écriture de la seconde bande lit le symbole blanc ($n \leq long(m)$), les deux têtes de lecture/écriture se déplacent vers la gauche et la machine passe dans l'état $q_6$ Positionnement sur le symbole d'indice $m$).

Sinon, toujours si un symbole $X$ est lu sur la première bande, si un symbole  $1$ est lu sur la seconde bande ($n > long(m)$), les deux têtes de lecture/écriture se déplacent vers la gauche et la machine passe dans l'état $q_7$ (Effaçage des symboles situés à gauche)

Dans le cas où \textbf{$n \leq long(m)$}, la machine se trouve dans l'état $q_6$, dans lequel les deux tête de lecture/écriture se déplacent vers la gauche tant que le symbole blanc est lu sur les deux bandes. Lorsqu'un symbole parmi $\{A, B, C, D\} $ est lu sur la première bande et qu'un symbole $1$ est lu sur la seconde bande (le symbole d'indice $n$ est atteint), les deux têtes de lecture/écriture se déplacent vers la gauche et la machine passe dans l'état $q_7$ (Effaçage des symboles situés à gauche du mot d'indice $n$).

A ce moment, peu importe si $n \leq long(m)$ ou si $n > long(m)$, on se trouve dans l'état $q_7$ et les deux tête de lecture/écriture sont situées au même niveau et de sorte que le symbole à renvoyer est situé à droite de la tête de lecture/écriture de la première bande. Dans cet état, tant que les tête de lecture/écriture des bandes 1 et 2 lisent respectivement un symbole parmi $\{A, B, C, D\}$ et un symbole 1, un symbole blanc est écrit sur la première bande et les deux tête de lecture/écriture se déplace à gauche. Si deux symboles blancs sont lus sur les deux bandes, alors les deux têtes de lecture/écriture se déplacent vers la droite et la machine passe dans l'état $q_8$ (Effaçage de la seconde bande).

Dans l'état $q_8$, tant qu'on lit un symbole $1$ sur la seconde bande, un symbole blanc est écrit sous la tête de lecture/écriture de la seconde bande et celle-ci de déplace à droite. Quand un symbole blanc est lu sous la tête de lecture/écriture de la seconde bande, celle-ci se déplace vers la gauche et la machine passe dans l'état $q_9$ (Déplacement de la tête de lecture/écriture de la première bande sur le symbole à renvoyer).

A l'état $q_9$, la première bande ne contient que le symbole à renvoyer des symboles blancs, et la seconde bande ne contient que des symboles blancs. Si $long(m) = 1$, la tête de lecture/écriture de la première bande est situé sur le symbole à renvoyer, sinon elle est située à sa gauche. Ainsi, celle-ci se déplace à droite tant qu'un symbole blanc est lu sur la première bande. Quand un symbole parmi $\{A, B, C, D, X\}$ est lu sur la première bande (le $n\up{ème}$ bit est atteint), la machine passe dans l'état final $q_f$.

\subsection{Preuve de l'algorithme}
Dans le cas où $n \leq long(m)$, il est évident qu'après la recopie des symboles constituant le nombre $n$ sur la seconde bande à partir du niveau du premier symbole du mot $m$, le dernier symbole $1$ de la seconde bande est au même niveau que le $n\up{ème}$ symbole du mot $m$ (c'est à dire le symbole à renvoyer). Il suffit alors de synchroniser les deux têtes de lecture/écriture et de les placer au niveau du premier symbole du mot $m$, puis d'avancer les deux têtes de lecture/écriture vers la droite jusqu'à ce qu'un symbole blanc soit lu sur la seconde bande pour détecter l'emplacement du symbole à renvoyer, ensuite il reste seulement à effacer les symboles autour de celui-ci puis d'effacer la seconde bande.
Le cas où $n > long(m)$ est détecté lors du déplacement des deux tête de lecture/écriture vers la droite de manière synchronisée, si on lit le symbole $X$ sur la première bande et le symbole $1$ sur la seconde. A ce moment, les deux tête de lecture/écriture se trouvent alors sur le symbole à renvoyer et comme tous les symboles $1$ de la seconde bande ont été consommés, il reste seulement à effacer tous les symboles à gauche de $X$ sur la première bande puis d'effacer la seconde bande.

\subsection{Analyse de la complexité}
Comme décrit dans la section 2.1.4, les séquences de transition entre deux états différents de la machine menant au résultat attendu sont $q_0 \rightarrow q_1 \rightarrow q_2 \rightarrow q_3 \rightarrow q_4  \rightarrow q_5 (\rightarrow q_6)? \rightarrow q_7 \rightarrow q_8 \rightarrow q_9 \rightarrow q_f$. Pour obtenir la fonction $T(N)$ mesurant le nombre de transition à emprunter pour aboutir au résultat en fonction de la longueur de la donnée $N$, il suffit de calculer le nombre de transition effectuées dans chacun de ces états en fonction de $N$ puis d'en faire la somme.

\begin{itemize}
\item Pour $q_0$ le nombre de transition est égal à $N - |s|$ avec $|s| \leq N - 2$ donc strictement positif.
\item Pour $q_1$ le nombre de transition est égal à $N - |P| + 1$ avec $|P| \leq N - 1$ donc strictement positif.
\item Pour $q_2$ le nombre de transition est égal à $N - |p|$ avec $|P| \leq N - 2$ donc strictement positif.
\item Pour $q_3$ le nombre de transition est égal à $N - |S|$ avec $|S| \leq N - 1$ donc strictement positif.
\item Pour $q_4$ le nombre de transition est égal à $N - |P| + 1$ donc strictement positif
\item Pour $q_5$ le nombre de transition est égal à $N - |s|$ qui est strictement positif
\item Pour $q_6$ le nombre de transition est égal à $|p| - |s| + 1$ avec $|p| > |s|$ car $n \leq long(m)$ (autrement on n'entrerait pas dans cet état), ce qui revient à $N - 2 * |s|$ qui est strictement positif
\item Pour $q_7$, le nombre de transition est égal à $N - |P| + 1$ si $n \leq long(m)$ ou $N - |S| + 1$ si $n > long(m)$ donc dans tous les cas $N - k + 1$ avec $k < N - 1$ qui est strictement positif
\item Pour $q_8$, le nombre de transition est égal à $N - |P| + 1$ qui est strictement positif
\item Pour $q_9$, le nombre de transition est égal à $N - |P|$ si $n \leq long(m)$ ou $N - |S|$ si $n > long(m)$ donc dans tous les cas $N - l + 1$ avec $l < N - 1$ qui est strictement positif
\item Pour $q_f$, le nombre de transition est égal à $0$
\end{itemize}

\vspace{5mm}

On en déduis la fonction $T(N) = 9 * N - |p| - 3 * |P| - 2 * |s| - |S| + 5 - k - l + (N - 2 * |s|)$ :

\vspace{5mm}

La définition de Big-O dit qu'une fonction f(x) est égale à O(g(x)) si il existe un facteur constant $c$ et un nombre réel $x_0$ tel que $|f(X)| \leq c * g(x)$.

\vspace{5mm}

Pour $x_0 = 0$ et $c = 1$, $|T(N)| \leq 9 * N + (N - 2 * |s|) + 5$ car $|p| < N$, $|P| < N$, $|s| < N$ et $|S| < N$ \\
Donc $T(N) = O(9 * N + (N - 2 * |s|))$

\vspace{5mm}

Pour $x_0 = 0$ et $c = 1$, $|T(N)| \leq 10 * N + 5$ car $N - 2 * |s|$ est strictement positif \\
Donc $T(N) = O(10 * N + 5)$

\vspace{5mm}

Pour $x_0 = 0$ et $c = 5$, $|T(N)| \leq 10 * N)$ \\
Donc $T(N) = O(10 * N)$

\vspace{5mm}

Pour $x_0 = 0$ et $c = 10$, $|T(N)| \leq N)$ \\
Donc $T(N) = O(N)$

\vspace{5mm}

On en déduis que la complexité de l'algorithme est dans l'ordre de $O(N)$.
